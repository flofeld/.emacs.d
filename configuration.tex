% Created 2019-09-22 So 15:27
% Intended LaTeX compiler: pdflatex
\documentclass{article}
\usepackage[utf8]{inputenc}
\usepackage[T1]{fontenc}
\usepackage{graphicx}
\usepackage{grffile}
\usepackage{longtable}
\usepackage{wrapfig}
\usepackage{rotating}
\usepackage[normalem]{ulem}
\usepackage{amsmath}
\usepackage{textcomp}
\usepackage{amssymb}
\usepackage{capt-of}
\usepackage{hyperref}
\usepackage{minted}
\author{Florian Feldmann}
\date{\today}
\title{Emacs Configuration}
\hypersetup{
 pdfauthor={Florian Feldmann},
 pdftitle={Emacs Configuration},
 pdfkeywords={},
 pdfsubject={},
 pdfcreator={Emacs 26.3 (Org mode 9.1.9)},
 pdflang={English}}
\begin{document}

\maketitle
\tableofcontents


\section{Configure \texttt{use-package}}
\label{sec:org453caf6}

I use \texttt{use-package} to install and configure my packages. My \texttt{init.el} includes
the initial setup for \texttt{package.el} and ensures that \texttt{use-package} is installed,
since I wanna do that right away.

This makes sure that \texttt{use-package} will install the package if it's not already
available. It also means that I should be able to open Emacs for the first time
on a fresh Debian box and have my whole environment automatically installed. I'm
not \emph{totally} sure about that, but we're gettin' close.

\begin{minted}[]{common-lisp}
(require 'use-package-ensure)
(setq use-package-always-ensure t)
\end{minted}

Always compile packages, and use the newest version available.

\begin{minted}[]{common-lisp}
(use-package auto-compile
  :config (auto-compile-on-load-mode))

(setq load-prefer-newer t)
\end{minted}

\section{Use sensible-defaults.el}
\label{sec:org29b7836}

Use \href{https://github.com/hrs/sensible-defaults.el}{sensible-defaults.el} for some basic settings.

\begin{minted}[]{common-lisp}
(load-file "~/.emacs.d/sensible-defaults.el/sensible-defaults.el")
(sensible-defaults/use-all-settings)
\end{minted}

\section{Set personal information}
\label{sec:orgfa0f412}

\begin{enumerate}
\item Who am I?
\label{sec:org10f41af}

\begin{minted}[]{common-lisp}
(setq user-full-name "Florian Feldmann"
	  user-mail-address "kontakt@flofeld.tech"
)
\end{minted}
\end{enumerate}

\section{Add \texttt{resources} to \texttt{load-path}}
\label{sec:orgb86ca34}

\begin{minted}[]{common-lisp}
(add-to-list 'load-path "~/.emacs.d/resources/")
\end{minted}


\section{Utility functions}
\label{sec:orgaed19cd}

Define a big ol' bunch of handy utility functions.

\begin{minted}[]{common-lisp}
(defun hrs/rename-file (new-name)
  (interactive "FNew name: ")
  (let ((filename (buffer-file-name)))
	(if filename
		(progn
		  (when (buffer-modified-p)
			 (save-buffer))
		  (rename-file filename new-name t)
		  (kill-buffer (current-buffer))
		  (find-file new-name)
		  (message "Renamed '%s' -> '%s'" filename new-name))
	  (message "Buffer '%s' isn't backed by a file!" (buffer-name)))))

(defun hrs/generate-scratch-buffer ()
  "Create and switch to a temporary scratch buffer with a random
	 name."
  (interactive)
  (switch-to-buffer (make-temp-name "scratch-")))

(defun hrs/kill-current-buffer ()
  "Kill the current buffer without prompting."
  (interactive)
  (kill-buffer (current-buffer)))

(defun hrs/visit-last-migration ()
  "Open the most recent Rails migration. Relies on projectile."
  (interactive)
  (let ((migrations
		 (directory-files
		  (expand-file-name "db/migrate" (projectile-project-root)) t)))
	(find-file (car (last migrations)))))

(defun hrs/add-auto-mode (mode &rest patterns)
  "Add entries to `auto-mode-alist' to use `MODE' for all given file `PATTERNS'."
  (dolist (pattern patterns)
	(add-to-list 'auto-mode-alist (cons pattern mode))))

(defun hrs/find-file-as-sudo ()
  (interactive)
  (let ((file-name (buffer-file-name)))
	(when file-name
	  (find-alternate-file (concat "/sudo::" file-name)))))

(defun hrs/region-or-word ()
  (if mark-active
	  (buffer-substring-no-properties (region-beginning)
									  (region-end))
	(thing-at-point 'word)))

(defun hrs/append-to-path (path)
  "Add a path both to the $PATH variable and to Emacs' exec-path."
  (setenv "PATH" (concat (getenv "PATH") ":" path))
  (add-to-list 'exec-path path))
\end{minted}

\section{UI preferences}
\label{sec:orgae8d529}
\begin{enumerate}
\item Tweak window chrome
\label{sec:orgf8b1ac3}

There's a tiny scroll bar that appears in the minibuffer window. This disables
that:

\begin{minted}[]{common-lisp}
(set-window-scroll-bars (minibuffer-window) nil nil)
\end{minted}

The default frame title isn't useful. This binds it to the name of the current
project:

\begin{minted}[]{common-lisp}
(setq frame-title-format '((:eval (projectile-project-name))))
\end{minted}

\item Use fancy lambdas
\label{sec:orga78c289}

Why not?

\begin{minted}[]{common-lisp}
(global-prettify-symbols-mode t)
\end{minted}

\item Load up a theme
\label{sec:orga727097}

I'm currently using the ``solarized-dark'' theme. I've got a scenic wallpaper, so
just a hint of transparency looks lovely and isn't distracting or hard to read.

\begin{minted}[]{common-lisp}
(use-package solarized-theme
  :config
  (load-theme 'solarized-dark t)

  (setq solarized-use-variable-pitch nil
		solarized-height-plus-1 1.0
		solarized-height-plus-2 1.0
		solarized-height-plus-3 1.0
		solarized-height-plus-4 1.0)

  (let ((line (face-attribute 'mode-line :underline)))
	(set-face-attribute 'mode-line          nil :overline   line)
	(set-face-attribute 'mode-line-inactive nil :overline   line)
	(set-face-attribute 'mode-line-inactive nil :underline  line)
	(set-face-attribute 'mode-line          nil :box        nil)
	(set-face-attribute 'mode-line-inactive nil :box        nil)
	(set-face-attribute 'mode-line-inactive nil :background "#f9f2d9")))



(defun hrs/apply-theme ()
  "Apply the `solarized-dark' theme and make frames just slightly transparent."
  (interactive)
  (load-theme 'solarized-dark t))
\end{minted}

If this code is being evaluated by \texttt{emacs -{}-daemon}, ensure that each subsequent
frame is themed appropriately.

\begin{minted}[]{common-lisp}
(if (daemonp)
	(add-hook 'after-make-frame-functions
			  (lambda (frame)
				(with-selected-frame frame (hrs/apply-theme))))
  (hrs/apply-theme))
\end{minted}

\item Use \texttt{moody} for a beautiful modeline
\label{sec:org10b6cd6}

This gives me a truly lovely ribbon-based modeline.

\begin{minted}[]{common-lisp}
(use-package moody
  :config
  (setq x-underline-at-descent-line t)
  (moody-replace-mode-line-buffer-identification)
  (moody-replace-vc-mode))
\end{minted}


\item Disable visual bell
\label{sec:org2150f22}

\texttt{sensible-defaults} replaces the audible bell with a visual one, but I really
don't even want that (and my Emacs/Mac pair renders it poorly). This disables
the bell altogether.

\begin{minted}[]{common-lisp}
(setq ring-bell-function 'ignore)
\end{minted}

\item Scroll conservatively
\label{sec:org1fa0b14}

When point goes outside the window, Emacs usually recenters the buffer point.
I'm not crazy about that. This changes scrolling behavior to only scroll as far
as point goes.

\begin{minted}[]{common-lisp}
(setq scroll-conservatively 100)
\end{minted}



\item Highlight the current line
\label{sec:org9b29de6}

\texttt{global-hl-line-mode} softly highlights the background color of the line
containing point. It makes it a bit easier to find point, and it's useful when
pairing or presenting code.

\begin{minted}[]{common-lisp}
(global-hl-line-mode)
\end{minted}

\item Highlight uncommitted changes
\label{sec:orgde0783d}

Use the \texttt{diff-hl} package to highlight changed-and-uncommitted lines when
programming.

\begin{minted}[]{common-lisp}
(use-package diff-hl
  :config
  (add-hook 'prog-mode-hook 'turn-on-diff-hl-mode)
  (add-hook 'vc-dir-mode-hook 'turn-on-diff-hl-mode))
\end{minted}
\end{enumerate}

\section{Project management}
\label{sec:org9030bd6}

I use a few packages in virtually every programming or writing environment to
manage the project, handle auto-completion, search for terms, and deal with
version control. That's all in here.

\begin{enumerate}
\item \texttt{ag}
\label{sec:org48d20e2}

Set up \texttt{ag} for displaying search results.

\begin{minted}[]{common-lisp}
(use-package ag)
\end{minted}

\item \texttt{company}
\label{sec:org546959c}

Use \texttt{company-mode} everywhere.

\begin{minted}[]{common-lisp}
(use-package company)
(add-hook 'after-init-hook 'global-company-mode)
\end{minted}

Use \texttt{M-/} for completion.

\begin{minted}[]{common-lisp}
(global-set-key (kbd "M-/") 'company-complete-common)
\end{minted}


\item \texttt{flycheck}
\label{sec:orgb07a4a5}

\begin{minted}[]{common-lisp}
(use-package flycheck)
\end{minted}
\end{enumerate}




\section{Programming environments}
\label{sec:orge415aef}

I like shallow indentation, but tabs are displayed as 8 characters by default.
This reduces that.

\begin{minted}[]{common-lisp}
(setq-default tab-width 4)
\end{minted}

Treating terms in CamelCase symbols as separate words makes editing a little
easier for me, so I like to use \texttt{subword-mode} everywhere.

\begin{minted}[]{common-lisp}
(use-package subword
  :config (global-subword-mode 1))
\end{minted}

Compilation output goes to the \texttt{*compilation*} buffer. I rarely have that window
selected, so the compilation output disappears past the bottom of the window.
This automatically scrolls the compilation window so I can always see the
output.

\begin{minted}[]{common-lisp}
(setq compilation-scroll-output t)
\end{minted}


\begin{enumerate}
\item CSS, Sass, and Less
\label{sec:org0d87fd8}

Indent by 2 spaces.

\begin{minted}[]{common-lisp}
(use-package css-mode
  :config
  (setq css-indent-offset 2))
\end{minted}

Don't compile the current SCSS file every time I save.

\begin{minted}[]{common-lisp}
(use-package scss-mode
  :config
  (setq scss-compile-at-save nil))
\end{minted}

Install Less.

\begin{minted}[]{common-lisp}
(use-package less-css-mode)
\end{minted}

\item Golang
\label{sec:org0ba394e}

Install \texttt{go-mode} and related packages:

\begin{minted}[]{common-lisp}
(use-package go-mode)
(use-package go-errcheck)
(use-package company-go)
\end{minted}

Define my \texttt{\$GOPATH} and tell Emacs where to find the Go binaries.

\begin{minted}[]{common-lisp}
(setenv "GOPATH" "/home/flofeld/go")
(hrs/append-to-path (concat (getenv "GOPATH") "/bin"))
\end{minted}

Run \texttt{goimports} on every file when saving, which formats the file and
automatically updates the list of imports. This requires that the \texttt{goimports}
binary be installed.

\begin{minted}[]{common-lisp}
(setq gofmt-command "goimports")
(add-hook 'before-save-hook 'gofmt-before-save)
\end{minted}

When I open a Go file,

\begin{itemize}
\item Start up \texttt{company-mode} with the Go backend. This requires that the \texttt{gocode}
binary is installed,
\item Redefine the default \texttt{compile} command to something Go-specific, and
\item Enable \texttt{flycheck}.
\end{itemize}

\begin{minted}[]{common-lisp}
(add-hook 'go-mode-hook
		  (lambda ()
			(set (make-local-variable 'company-backends)
				 '(company-go))
			(company-mode)
			(if (not (string-match "go" compile-command))
				(set (make-local-variable 'compile-command)
					 "go build -v && go test -v && go vet"))
			(flycheck-mode)))
\end{minted}

Config for company-go

\begin{minted}[]{common-lisp}
(when (memq window-system '(mac ns x))
  (exec-path-from-shell-initialize))

;;For company-go
(require 'company)
(require 'company-go)

(use-package company
  :defer 2
  :diminish
  :custom
  (company-begin-commands '(self-insert-command))
  (company-idle-delay .1)
  (company-minimum-prefix-length 2)
  (company-show-numbers t)
  (company-tooltip-align-annotations 't)
  (company-tooltip-limit 20)
  (company-begin-commands '(self-insert-command))
  (global-company-mode t))

(add-hook 'go-mode-hook
	  (lambda ()
		(set (make-local-variable 'company-backends) '(company-go))
		(company-mode)))

\end{minted}



\item JavaScript and CoffeeScript
\label{sec:orgba8174f}

Install \texttt{coffee-mode} from editing CoffeeScript code.

\begin{minted}[]{common-lisp}
(use-package coffee-mode)
\end{minted}

Indent everything by 2 spaces.

\begin{minted}[]{common-lisp}
(setq js-indent-level 2)

(add-hook 'coffee-mode-hook
		  (lambda ()
			(yas-minor-mode 1)
			(setq coffee-tab-width 2)))
\end{minted}

\item Lisps
\label{sec:org19dc91a}

I like to use \texttt{paredit} in Lisp modes to balance parentheses (and more!).

\begin{minted}[]{common-lisp}
(use-package paredit)
\end{minted}

\texttt{rainbow-delimiters} is convenient for coloring matching parentheses.

\begin{minted}[]{common-lisp}
(use-package rainbow-delimiters)
\end{minted}

All the lisps have some shared features, so we want to do the same things for
all of them. That includes using \texttt{paredit}, \texttt{rainbow-delimiters}, and
highlighting the whole expression when point is on a parenthesis.

\begin{minted}[]{common-lisp}
(setq lispy-mode-hooks
	  '(clojure-mode-hook
		emacs-lisp-mode-hook
		lisp-mode-hook
		scheme-mode-hook))

(dolist (hook lispy-mode-hooks)
  (add-hook hook (lambda ()
				   (setq show-paren-style 'expression)
				   (paredit-mode)
				   (rainbow-delimiters-mode))))
\end{minted}

If I'm writing in Emacs lisp I'd like to use \texttt{eldoc-mode} to display
documentation.

\begin{minted}[]{common-lisp}
(use-package eldoc
  :config
  (add-hook 'emacs-lisp-mode-hook 'eldoc-mode))
\end{minted}

I also like using \texttt{flycheck-package} to ensure that my Elisp packages are
correctly formatted.

\begin{minted}[]{common-lisp}
(use-package flycheck-package)

(eval-after-load 'flycheck
  '(flycheck-package-setup))
\end{minted}
\end{enumerate}








\section{Publishing and task management with Org-mode}
\label{sec:org301659a}

\begin{minted}[]{common-lisp}
(use-package org)
\end{minted}

\begin{enumerate}
\item Display preferences
\label{sec:org457e5d7}

I like to see an outline of pretty bullets instead of a list of asterisks.

\begin{minted}[]{common-lisp}
(use-package org-bullets
  :init
  (add-hook 'org-mode-hook 'org-bullets-mode))
\end{minted}

I like seeing a little downward-pointing arrow instead of the usual ellipsis
(\texttt{...}) that org displays when there's stuff under a header.

\begin{minted}[]{common-lisp}
(setq org-ellipsis "⤵")
\end{minted}

Use syntax highlighting in source blocks while editing.

\begin{minted}[]{common-lisp}
(setq org-src-fontify-natively t)
\end{minted}

Make TAB act as if it were issued in a buffer of the language's major mode.

\begin{minted}[]{common-lisp}
(setq org-src-tab-acts-natively t)
\end{minted}

When editing a code snippet, use the current window rather than popping open a
new one (which shows the same information).

\begin{minted}[]{common-lisp}
(setq org-src-window-setup 'current-window)
\end{minted}



\item Task and org-capture management
\label{sec:org9b60fd8}

Record the time that a todo was archived.

\begin{minted}[]{common-lisp}
(setq org-log-done 'time)
\end{minted}



\item Exporting
\label{sec:orgc83fa56}

Allow export to markdown and beamer (for presentations).

\begin{minted}[]{common-lisp}
(require 'ox-md)
(require 'ox-beamer)
\end{minted}


Translate regular ol' straight quotes to typographically-correct curly quotes
when exporting.

\begin{minted}[]{common-lisp}
(setq org-export-with-smart-quotes t)
\end{minted}

\begin{enumerate}
\item Exporting to HTML
\label{sec:orge8721c6}

Don't include a footer with my contact and publishing information at the bottom
of every exported HTML document.

\begin{minted}[]{common-lisp}
(setq org-html-postamble nil)
\end{minted}



\item Exporting to PDF
\label{sec:orgdf082f3}

enable latex

\begin{minted}[]{common-lisp}

(require 'ox-latex)
(unless (boundp 'org-latex-classes)
  (setq org-latex-classes nil))
(add-to-list 'org-latex-classes
			 '("article"
			   "\\documentclass{article}"
			   ("\\section{%s}" . "\\section*{%s}")))

\end{minted}



I want to produce PDFs with syntax highlighting in the code. The best way to do
that seems to be with the \texttt{minted} package, but that package shells out to
\texttt{pygments} to do the actual work. \texttt{pdflatex} usually disallows shell commands;
this enables that.

\begin{minted}[]{common-lisp}
(setq org-latex-pdf-process
	  '("xelatex -shell-escape -interaction nonstopmode -output-directory %o %f"
		"xelatex -shell-escape -interaction nonstopmode -output-directory %o %f"
		"xelatex -shell-escape -interaction nonstopmode -output-directory %o %f"))
\end{minted}

Include the \texttt{minted} package in all of my \LaTeX{} exports.

\begin{minted}[]{common-lisp}
(add-to-list 'org-latex-packages-alist '("" "minted"))
(setq org-latex-listings 'minted)
\end{minted}
\end{enumerate}



\item \TeX{} configuration
\label{sec:orgabeeded}

I rarely write \LaTeX{} directly any more, but I often export through it with
org-mode, so I'm keeping them together.

Automatically parse the file after loading it.

\begin{minted}[]{common-lisp}
(setq TeX-parse-self t)
\end{minted}

Always use \texttt{pdflatex} when compiling \LaTeX{} documents. I don't really have any
use for DVIs.

\begin{minted}[]{common-lisp}
(setq TeX-PDF-mode t)
\end{minted}

Open compiled PDFs in \texttt{okular} instead of in the editor.

\begin{minted}[]{common-lisp}
(add-hook 'org-mode-hook
	  '(lambda ()
		 (delete '("\\.pdf\\'" . default) org-file-apps)
		 (add-to-list 'org-file-apps '("\\.pdf\\'" . "okular %s"))))
\end{minted}




\item Wrap paragraphs automatically
\label{sec:org0d03931}

\texttt{AutoFillMode} automatically wraps paragraphs, kinda like hitting \texttt{M-q}. I wrap
a lot of paragraphs, so this automatically wraps 'em when I'm writing text,
Markdown, or Org.

\begin{minted}[]{common-lisp}
(add-hook 'text-mode-hook 'auto-fill-mode)
(add-hook 'gfm-mode-hook 'auto-fill-mode)
(add-hook 'org-mode-hook 'auto-fill-mode)
\end{minted}
\end{enumerate}






\section{Editing settings}
\label{sec:orgcf74d3f}



\begin{enumerate}
\item Look for executables in \texttt{/usr/local/bin}
\label{sec:orgc378ebe}

\begin{minted}[]{common-lisp}
(hrs/append-to-path "/usr/local/bin")
\end{minted}

\item Save my location within a file
\label{sec:org85e3537}

Using \texttt{save-place-mode} saves the location of point for every file I visit. If I
close the file or close the editor, then later re-open it, point will be at the
last place I visited.

\begin{minted}[]{common-lisp}
(save-place-mode t)
\end{minted}






\item Switch and rebalance windows when splitting
\label{sec:org468f173}

When splitting a window, I invariably want to switch to the new window. This
makes that automatic.

\begin{minted}[]{common-lisp}
(defun hrs/split-window-below-and-switch ()
  "Split the window horizontally, then switch to the new pane."
  (interactive)
  (split-window-below)
  (balance-windows)
  (other-window 1))

(defun hrs/split-window-right-and-switch ()
  "Split the window vertically, then switch to the new pane."
  (interactive)
  (split-window-right)
  (balance-windows)
  (other-window 1))

(global-set-key (kbd "C-x 2") 'hrs/split-window-below-and-switch)
(global-set-key (kbd "C-x 3") 'hrs/split-window-right-and-switch)
\end{minted}
\end{enumerate}
\end{document}
